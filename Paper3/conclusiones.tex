\section{Conclusiones}

Se obtuvo en forma íntegra un sistema óptico de captura de movimiento basado en marcadores, que a partir de las capturas de video de una persona en un ambiente de laboratorio con las condiciones adecuadas, obtiene la posición 3D de  los marcadores presentes en el cuerpo de dicha persona, logrando representar su movimiento con una precisión del orden del centímetro.

Si bien inicialmente el objetivo era que el sistema funcionara por lo menos para el caso de uso de la marcha, se han probado otros  movimientos con resultados aceptables. La aplicación desarrollada permite a partir de múltiples capturas de vídeo de un sujeto en movimiento, detectar los marcadores en cada toma de video. Junto a la información de las cámaras, luego reconstruye la posición de los marcadores en el espacio y finalmente logra identificar cada marcador a lo largo de la secuencia temporal. Cabe destacar que el sistema implementado no es solo óptico, sino que es lo bastante general para funcionar con cualquier sistema de adquisición que genere imágenes, por ejemplo con imágenes infrarrojas provenientes de sistemas de captura modernos.