
\begin{abstract}
%\boldmath
En este artículo se propone un sistema óptico de captura de movimiento basado en marcadores que facilita la tarea en el análisis biomecánico del movimiento de las personas. Dicho sistema se compone de bloques independientes, dando una solución general que posibilita modificar o sustituir sus componentes fácilmente. Se describen lineamientos para generar un laboratorio virtual a partir del cual obtener una base de datos con secuencias de videos sintéticas y se exploran las características que debe tener un laboratorio de captura óptico basado en marcadores para facilitar el procesamiento. Se reune un conjunto de métricas para medir la performance de cada bloque y del sistema en su totalidad. Las pruebas realizadas sobre el software implementado reflejaron que el mismo tiene una precisión del orden del centímetro sobre secuencias obtenidas en ambientes controlados. Estos resultados son buenos teniendo en cuenta que los algoritmos utilizados en cada bloque son de complejidad baja y se pueden optimizar.
\keywords{
	Biomecánica, calibración, detección de marcadores, reconstrucción, seguimiento, laboratorio virtual.
}
\end{abstract}


%La propuesta inicial fue realizada por investigadores de biomecánica del Departamento de Biofísica de la Facultad de Medicina de la Universidad de la República, Uruguay, en busca de una herramienta de código abierto que le permita obtener datos y estadísticas específicas que las herramientas existentes no pueden ofrecer.