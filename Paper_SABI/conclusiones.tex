\section{Conclusiones}

Se obtuvo en forma íntegra un sistema óptico de captura de movimiento basado en marcadores, que a partir de las capturas de video de una persona en un ambiente de laboratorio con las condiciones adecuadas, obtiene la posición 3D de  los marcadores presentes en el cuerpo de dicha persona, logrando representar su movimiento con una precisión del orden del centímetro.
%
%Si bien inicialmente el objetivo fue que el sistema funcionara por lo menos para el caso de uso de la marcha, se han probado otros  movimientos con resultados aceptables. La aplicación desarrollada permite a partir de múltiples capturas de vídeo de un sujeto en movimiento, detectar los marcadores en cada toma de video. Junto a la información de las cámaras, luego reconstruye la posición de los marcadores en el espacio y finalmente logra identificar cada marcador a lo largo de la secuencia temporal. Cabe destacar que el sistema implementado no es solo óptico, sino que es lo bastante general para funcionar con cualquier sistema de adquisición que genere imágenes, por ejemplo con imágenes infrarrojas provenientes de sistemas de captura modernos.

Por otro lado, la implementación separa cada etapa del proceso en módulos distintos, capaces de funcionar de manera independiente. Lo cual permite que el sistema no sea estrictamente óptico sino lo bastante general como para funcionar con cualquier sistema de adquisición que genere imágenes. %Esto además permite aumentar la reproductibilidad en el área, puse se cuenta con un sistema completo y esrtructurado donde fácilmente se pueden modificar, ingresar y probar distintas partes comparando sus desempeños.\\

Al realizar pruebas con secuencias reales  provenientes de tres cámaras en un laboratorio fuera de las hipótesis de captura, se producen problemas en las etapas de segmentación y reconstrucción. Efectuando extracción de fondo y modificaciones sobre la reconstrucción se logran mejores resultados aunque apenas aceptables. Constatando el fuerte impacto que tiene una pobre metodología de captura sobre el posterior procesamiento.

Se logra contribuir con reproductibilidad y metodología de diseño en el área a través de un sistema completo y estructurado de captura de movimiento.
%Para comparar los desempeños se definieron una serie de métricas, mediante las cuales se obtuvo una medida de performance de las etapas del sistema y del sistema completo, de forma tal de tener valores de referencia sobre los cuales comparar futuras versiones u otros sistemas.