
\begin{abstract}
%\boldmath
This paper proposes an optical motion capture system based on markers that facilitates biomechanical analysis from people movements.
The system consists of separate blocks, giving an overall solution which allows to easily modify or replace components.
Guidelines are described for generating a virtual laboratory from which to obtain a database with sequences of synthetic videos and are explored the characteristics required of a laboratory based on optical capture markers to facilitate processing.
A set of metrics is presented to measure the performance of each block and the overall system.
Tests on the software implemented reflected that it has an accuracy in the order of centimeters on sequences obtained in controlled environments. These results are acceptables considering that the algorithms used in each block are of low complexity and can be optimized.
\keywords{
	Biomechanic, calibration, segmentation, reconstruction, tracking, virtual laboratory.
}
\end{abstract}

