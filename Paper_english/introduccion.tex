\section{Introduction}

The video analysis is an essential tool for study and data collection. The tracking of markers can be used to calculate position and other associated variables such as speed, acceleration and therefore movement.
The manual tracking of those points can be tedious so it is necessary to have a tool to perform this task automatically. 
Some examples that illustrate these needs are: at \emph{health care} in the area of \emph{physiotherapy}, \emph{biomechanics academic research},
\emph{measures of performance} in professional sports and \emph{3D animation}.
These examples define use cases with different characteristics, so the search for a single solution that covers particular needs of all of them is a complex issue. To have these tools is essential for professional needs, where other alternatives are high-cost commercial products.\\
\hspace*{0.5cm}This project sought to perform a basic and functional open source video analysis application to provide a solution to the above mentioned needs developing a prototype software that covers the entire problem generally and which can be extended to other use cases.\\
\hspace*{0.5cm}The system created aims, under certain controlled conditions, to obtain the spatial coordinates of a number of points on a subject. To do this the subject, with a black suit and white markers, is placed in an environment with adequate lighting, then filmed with multiple cameras over time. This information is transfered to the computer and processed to obtain the 3D trayectories of all markers.\\
\hspace*{0.5cm}Working from the algorithm proposed by Herda \textit{et al.} \cite{herda}, it is proposed a reproducible version from it, containing measures of performance on each stage, as well as a methodology for designing a virtual laboratory.

