\section{Introduction}

The Video analysis is a critical tool for data collection and study. Tracking reference points are used for the position calculation and other associated variables such as speed, acceleration and hence movement.
The manual tracking of those points can be tedious so it is necessary to have a tool to perform this task automatically. 
Some examples that illustrate these needs are in \emph{health care level} in the area of \emph{physiotherapy}, \emph{biomechanics academic research},
\emph{measures of performance} in top-level sport and the \emph{3D animation}, among others.
The examples above defined use cases with different characteristics, so the search for a single solution covering the particular needs of all of them is complex. Having these tools is critical to the needs of professional teams, whose alternative are high-cost commercial products.\\
\hspace*{0.5cm}This project sought to perform a basic and functional open source application video analysis, to provide solution to the above needs a project basis using existing free software, or failing that, developing a prototype software that covers the entire problem generally, for extending the application to other use cases.\\
\hspace*{0.5cm}The system created aims under certain controlled conditions to obtain the spatial coordinates of a number of interest points on a patient.
Placing the patient with black suit and white markers in an environment with adequate lighting, is filmed with multiple cameras over time, this information is brought to the computer and processed to obtain the 3D trayectories of the markers.\\
\hspace*{0.5cm}Working from the algorithm proposed by Herda \cite{herda}, trying to do reproducible version thereof containing measures performance on each stage,  as well as a methodology for designing a virtual laboratory jointly developing a complete test system.

