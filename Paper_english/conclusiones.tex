\section{Conclusions}

An optical motion capture system based on markers was obtained completely,
from videos captured from a person in a lab environment with controlled conditions, it manages to obtain 3D position of markers in the body of the person, obtaining movement with accuracy in order of centimeters.\\
\hspace*{0.5cm} Furthermore, this implementation separates each process step in different modules capable of operating independently, allowing to work with any acquisition system that generates pictures to analyse.\\
\hspace*{0.5cm} When testing with real sequences from three cameras in a laboratory outside the motion capture hypothesis, problems occur in the stages of segmentation and reconstruction. Performing background subtraction on sequences and modifications on reconstruction block can achieve better results although barely acceptable. This verifies how strong the effect a poor methodology of captured data can have in further steps in the motion capture system.\\
\hspace*{0.5cm}This work does contribute on reproducibility and design methodology in the area through a structured full motion capture system.