\section{State of the art}\label{estadoDelArte}

We have focused our work on developing a system with acceptable accuracy but also affordable, so we direct the search towards systems which uses markers and normal cameras. While it was not found much literature reviewing the state of the art of such systems, we did find articles about full or partial implementations of this kind of systems. Within this bibliography there is a thesis produced by Herda \textit{et al.} \cite{herda}. This was used as a base for the development of our system because the thesis is quoted on many articles and the system described has the features which we were looking for. However, the explanation of some processes in this thesis has some ambiguities.

We have also searched for currently available systems. The commercial motion capture systems use mostly infrared cameras and are known for their good performance but many of them are very expensive. We haven't found free software alternatives that perform all stages of a motion capture system but there are some tools available that allow to perform some of these stages separately.

In order to develop and test our system we searched for video sequences of motion capture. Many databases were found but they were all ruled out for not conforming to the assumptions made in this paper. These databases were designed for making motion captures with infrared cameras so the laboratory conditions are not adequate for normal cameras. However these databases were used to generate a virtual laboratory which is described in section \ref{section_base_de_datos}.



%%%	 VERSION  ESPAÑOL %%%%%%%%%%%%%%%%%%%%%%%%%%%%%%%%%%%%%%%%%%
%
%\section{Estado del arte}\label{estadoDelArte}
%% deberia quedar claro previo a esto que la motivacion surgio del clinicas, que utilizaban el dvideow y que les daba problemas.
%%Porque se utilizaron sistemas con marcadores respecto a sistema sin marcadores u otras alternativas.
%%Software bibliografia y base de datos de de sistemas con marcadores
%% como no se encontraron soluciones que cumplieran nuestros requisitos se opto por implementarlas nosotros
%% antes de la busqueda del software y la base de datos sabiamos que ibamos a usar captura con marcadores y camaras opticas convencionales.
%%Para el desarrollo del sistema se relevó la bibliografía disponible así como los sistemas de captura de movimiento e implementaciones de software existentes. Dada la necesidad de disponer de secuencias de video tanto para el desarrollo del sistema como para probar su desempeño se relevaron las bases de secuencias de video disponibles.%
%%Respecto a la búsqueda de bibliografía se han encontrado diversos artículos que hacen una revisión de las distintas técnicas de captura de movimiento. Dado que nuestra búsqueda estaba centrada 
%Centrados en el desarrollo de un sistema con precisión aceptable pero que además fuera accesible, orientamos la búsqueda hacia sistemas que emplean cámaras convencionales y marcadores. Si bien no se ha encontrado mucha bibliografía que resumiera el estado del arte de dichos sistemas sí se han encontrado diversos artículos que describen implementaciones, totales o parciales, que utilizan esta metodología. Dentro de esa bibliografía se encontró la tesis elaborada por Herda  \cite{herda}. La misma se utilizó como base para nuestro desarrollo dado que el sistema que propone posee las características buscadas y es ampliamente citado. No obstante, dicha tesis presenta ambigüedades en la descripción de la metodología utilizada en algunos procesos.\\
%% se puede mencionar que los sistemas de marcadores tienen mayor evidencia de que tiene mayor precision. Se podría citar algun artículo.
%% se debería citar algun articulo (review) sobre el sistema de captura de movimiento que utliza marcadores y camaras opticas.
%%También se analizaron las soluciones que se encuentran actualmente disponibles.
%\hspace*{0.5cm}En cuanto a las soluciones disponibles actualmente, por un lado se tienen sistemas de captura de movimiento comerciales que destacan por su performance. La mayoría de estos sistemas utilizan cámaras infrarrojas, lo que permite una detección más eficaz de marcadores pero con la desventaja de su elevado costo. Existen también alternativas de software libre, aunque no se han encontrado implementaciones que realicen todas las etapas de un sistema de captura de movimiento en forma íntegra, sí existen implementaciones que permiten realizar algunas etapas de manera aislada.\\ %
%% de un sistema de captura como toolbox en \textit{ Matlab} o librerías como \textit{OpenCV}. Alguna de estas herramientas fueron utilizadas en el sistema desarrollado.
%%%%%
%%Por otra parte se realizó una búsqueda de secuencias de capturas de movimiento. 
%\hspace*{0.5cm}En busca de secuencias de captura de movimiento, se encontraron numerosas bases de datos, estas se descartaron por no ajustarse a las hipótesis planteadas en este trabajo. Dichas bases fueron diseñadas para capturas con cámaras infrarrojas por lo que las condiciones de laboratorio empleadas no son las adecuadas para cámaras convencionales. No obstante dichas bases fueron utilizadas para generar un laboratorio de captura virtual el cual que se describe en  la sección \ref{section_base_de_datos}.
%
%
%%%%%%%%%%%%%%%%%%%%%%%%%%%%%%%%%%%%%%%%%%%%%%%%%%%%%%%
%
%


