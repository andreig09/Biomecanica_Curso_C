
\begin{abstract}
%\boldmath
This paper proposes an optical motion capture system based on markers that facilitates human biomechanical analysis. The system consists of separate blocks, giving an overall solution which allows to easily modify or replace components. Guidelines are described for generating a virtual laboratory to obtain a database with sequences of synthetic videos and explore the characteristics required for a laboratory based on optical motion capture system. A set of metrics is presented to measure the performance of each block and the overall system. Tests on the implemented software  reflected that it has an accuracy in the order of centimeters on sequences obtained in controlled environments. These results are acceptable considering that the algorithms used in each block are of low complexity and can be optimized.
%\keywords{
%	Biomechanic, calibration, segmentation, reconstruction, tracking, virtual laboratory.
%}
\end{abstract}

