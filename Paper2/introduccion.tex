\section{Introducción}
Especialistas de distintos ámbitos académicos o profesionales se encuentran habitualmente en la necesidad de realizar estudios del movimiento del cuerpo humano. Esta tarea implica registrar la posición de miembros o articulaciones en el espacio y su correspondiente evolución en el tiempo.
 

Algunos ejemplos correspondientes a distintas áreas que ilustran estas necesidades son a \emph{nivel asistencial} en el área de \emph{fisioterapia}, \emph{investigación académica en biomecánica,} \emph{medidas de performance} en el deporte de alto nivel y \emph{Animación 3D}, entre otras. 


En este contexto, el análisis de video  es  una  herramienta  fundamental para la recolección y estudio de datos. El seguimiento de puntos de referencia se utiliza para el cálculo de posición y otras variables asociadas como son la velocidad, la aceleración y por ende desplazamientos. Trabajar con video permite además estudiar secuencialmente situaciones estáticas en el tiempo. El seguimiento de dichos puntos resultaría tedioso si se hiciera manualmente por lo que resulta necesario contar con una herramienta que realice esta tarea automáticamente.


Los ejemplos mencionados anteriormente definen distintos casos de uso con características disímiles, de manera que la búsqueda de una solución única que abarque las necesidades particulares de todos ellos resulta compleja. Por ejemplo, en el ámbito deportivo la velocidad del movimiento es una variable importante a tener en cuenta para desarrollar una solución, de esta variable depende la elección tanto del sistema de adquisición como de los algoritmos más eficaces para el registro del movimiento.  De igual manera definir la portabilidad del sistema depende de si la actividad a relevar es en condiciones de laboratorio controladas o al aire libre, debido a la protección y transporte de equipos o las variaciones en las condiciones de iluminación, ruido, etc..



Al  día  de  hoy, las  soluciones  de  software  disponibles que  podrían  asistir al especialista en su tarea, son mayormente comerciales. Las pocas alternativas de código abierto, carecen de las características necesarias para el especialista o están enfocadas hacia otras áreas de aplicación. Contar con este tipo de herramientas es fundamental para las necesidades de los equipos de profesionales, cuya alternativa son productos comerciales de alto costo.



QUIZAS ACÁ REDONDEAR CON LO QUE SE PRETENDE EN EL PAPER, O SEA LOS OBJETIVOS PROPUESTOS Y LUEGO LOS REQUERIMIENTOS QUE SE TIENEN.
