\section{Introducción}
El análisis de video  es  una  herramienta  fundamental para la recolección y estudio de datos. El seguimiento de puntos de referencia se utiliza para el cálculo de posición y otras variables asociadas como son la velocidad, la aceleración y por ende desplazamientos. Trabajar con video permite además estudiar secuencialmente situaciones estáticas en el tiempo. El seguimiento de dichos puntos resultaría tedioso si se hiciera manualmente por lo que resulta necesario contar con una herramienta que realice esta tarea automáticamente.
Algunos ejemplos correspondientes a distintas áreas que ilustran estas necesidades son a \emph{nivel asistencial} en el área de \emph{fisioterapia}, \emph{investigación académica en biomecánica,} \emph{medidas de performance} en el deporte de alto nivel y \emph{Animación 3D}, entre otras. 
Los ejemplos mencionados anteriormente definen distintos casos de uso con características disímiles, de manera que la búsqueda de una solución única que abarque las necesidades particulares de todos ellos resulta compleja. Contar con este tipo de herramientas es fundamental para las necesidades de los equipos de profesionales, cuya alternativa son productos comerciales de alto costo.

En virtud de estas necesidades, este proyecto buscó realizar una aplicación básica y funcional de código abierto de análisis de video, que proporcione solución a las necesidades que se describieron ya sea utilizando como base algún  proyecto  de  software  libre  existente, o en su defecto, desarrollando un prototipo de software básico completo que abarque el problema en forma general, para luego estudiar extender la aplicación hacia otros casos de uso.

El sistema creado pretende bajo ciertas condiciones controladas, obtener las coordenadas espaciales de un número de puntos de interés sobre un paciente. Una de las formas de obtener esto, y la estudiada en este trabajo, es colocar al sujeto con traje negro con marcadores en un ambiente con iluminación adecuada, filmar con varias cámaras a lo largo del tiempo, adquirir esta información en la computadora y mediante un posterior procesamiento obtener la posición 3D de cada uno de los puntos de interés (los marcadores) a lo largo de toda la secuencia.
