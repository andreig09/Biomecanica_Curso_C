\section{Performance}

Fueron utilizadas las métricas establecidas en HumanEva \cite{humaneva} para comparar los conjuntos de datos obtenidos a la salida de cada bloque individual con aquel conjunto de referencia obtenido en el ground truth de base de datos, buscando primero la correspondencia entre puntos y luego la distancia euclidiana (en 2D para cámaras y 3D para espacio reconstruido) entre los puntos de ambos conjuntos.


Son generadas múltiples secuencias de vídeo desde varias vistas en el espacio en el laboratorio virtual ademas de los parámetros de las cámaras sintéticas.


Estas secuencias sintéticas van mas allá del caso ideal y contemplan problemas presentes en la realidad siendo el principal ejemplo que ciertos marcadores se encuentran ocultos por el propio cuerpo del sujeto a capturar. 


El error de detección de los marcadores en cada una de las cámaras no supera un par de pixeles para el caso de cámaras con resolución en imagen de 1600x600. Es posible reducir la resolución de las cámaras hasta 800x300 manteniendo los mismos resultados, pero en resoluciones menores comienza a degradarse la tasa de detección de marcadores en cámaras individuales lo cual perjudica la siguiente etapa de reconstrucción. Fueron realizadas pruebas inyectando ruido en el bloque de detección y su impacto en etapas posteriores encontrando que se puede trabajar con un poco mas de ruido (no mas de tres pixeles de error) sin comprometer significativamente el error final.


Si las cámaras se encuentran en las condiciones de error anteriores, una cobertura total con 17 cámaras rodeando el movimiento a capturar permite reconstruir todas las trayectorias con un error por debajo del centímetro con errores máximos cercanos a tres centímetros. Los resultados se mantienen dentro de estos limites si se reduce a ocho cámaras colocando pares de cámaras en cada esquina del espacio de captura. Reducir aún más el conjunto de cámaras sin degradar la performance es posible con ciertas condiciones: un conjunto de seis cámaras colocadas en dos costados de a tres, y cada costado tiene dos cámaras próximas con una tercera mas alejada. De esta manera se tiene las condiciones mínimas por las cuales se estableció la reconstrucción, tener dos cámaras para reconstruir y una tercera para confirmar.


Este caso limite de reconstrucción muestra comportamiento global similares a los anteriores, pero muestra degradaciones puntuales para ciertas trayectorias de marcadores que superan los limites aceptables.

