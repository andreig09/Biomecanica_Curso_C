\section{Base de datos}


\subsection{Introducción}
\label{}
Mencionar que tipo de base de datos se necesita encontrar\\
Con el fin de relevar distinta clases de actividades humanas y en particular la marcha, se pretende trabajar con una base de datos basada en marcadores, que contenga un cierto número de sujetos, realizando una variedad de movimientos predefinidos, donde para cada sujeto a estudiar, se tengan múltiples secuencias de videos 2D del movimiento obtenidas a partir de cámaras situadas en un entorno 3D cerrado previamente acondicionado. Dicha base debe contar con el correspondiente ground truth 2D y 3D de los datos de movimiento relevados, en nuestro caso como nuestro sistema está basado en marcadores, dicha información consiste en obtener las coordenadas espaciales a lo largo del tiempo para cada marcador de interés. Así como la información de calibración.blablabla MEJORAR \\

Mencionar porque se tuvo que generar una y que contiene a grandes rasgos.\\
Se propone una estructura de datos para almacenar la información relevante del ground truth y  se cuenta con código de soporte que facilita la generación de nuevas secuencias así como la gestión de la información en la estructura de datos, evaluar resultados y realizar el  posterior análisis de performance.


blablablabl
Un ambiente controlado con cámaras ajustadas convenientemente provee un escenario ideal para obtener información óptica de los marcadores. El sujeto es condicionado a moverse sobre un espacio de captura determinado con una vestimenta particular; la elección adecuada de estos parámetros junto con las condiciones de iluminación y el tipo de fondo facilitan la recolección de información, el posterior análisis y sobre todo la performance del sistema completo.

\subsection{Revisión de base de datos}
\label{}
En esta sección se muestra  resultados del relevamiento de bases de datos útiles para el análisis de la marcha.
Inicialmente nuestro objetivo era obtener un sistema funcional para el caso de la marcha humana, aunque luego de la implementación del mismo se generalizaron sus usos. Por lo tanto bases de datos con este tipo de movimiento fueron el eje central de la búsqueda.

Se encontraron principalmente tres páginas web que a manera de compendio agrupan  bases de datos útiles en varias de las ramas de la visión por computadora.\\

\hspace{-0.7cm} \textbf{CVonline \footnote{\textcolor{blue}{\underline{\url{http://homepages.inf.ed.ac.uk/rbf/CVonline}}}. Accedido 30-11-14.} } 

Bastante completa, no solo contiene enlaces a  varias bases de datos sino reune bibliografía e implementaciones útiles.\\


\hspace{-0.7cm} \textbf{Computer Vision Papers \footnote{\textcolor{blue}{\underline{\url{ http://www.cvpapers.com/datasets.html}}}. Accedido 30-11-14.} } 

	 Solo reúne enlaces de bases de datos.\\
		

\hspace{-0.7cm} \textbf{Yet Another Computer Vision Index To Datasets (YACVID)} \footnote{\textcolor{blue}{\underline{\url{http://riemenschneider.hayko.at/vision/dataset/}}}. Accedido 30-11-14. } 	

Una característica importante es que la página contiene direcciones a bases de datos relativamente nuevas y resalta aquellas que son usadas con mayor frecuencia. \\
	

Dentro de estas páginas se encuentra una gran variedad de bases de datos que tratan el caso particular de la marcha, en la tabla \ref{bases_relevadas} se muestran algunas de las bases relevadas y sus características representativas, que permiten hacer una idea del panorama global encontrado a la hora de recopilar información en la web.  

\begin{table}[h!]
	\centering	
	\caption{Comparación de algunas bases de datos disponibles y empleadas por la comunidad.}
	\label{bases_relevadas}
	\begin{minipage}{\textwidth} %por algún motivo el arabic no funciona, no me pone llamados a pie de página numericos	
	\begin{tabular}{||l|ccccc||} 
\rowcolor[HTML]{CBCEFB} 
\hline
\textbf{Base}     & \textbf{Cantidad }  & \textbf{Nro. de }   & \textbf{Entorno} & \textbf{Número de} & \textbf{Calibración}\\
\rowcolor[HTML]{CBCEFB} 
\textbf{de datos} & \textbf{de sujetos} & \textbf{secuencias} &         & \textbf{cámaras }  &  \textbf{disponible} \\


\hline \hline
M.C.L.\footnote{Motion Capture Lab}  & 3 		& 		299	   & Interior&     1    &    No      \\ \hline
C.M.U  \footnote{Carnegie Mellon University Motion Capture Database}	
	   & >100     &       2605   & Interior&      1   &    No       \\ \hline
G.T \footnote{Georgia Tech} &       20    & $\sim$100           & Interior y &   3      &  Si       \\ 
		 &		 &					 & exterior        &         &    \\ \hline
U.S. \footnote{University of Southampton Database} &       >100    &     $\sim$       & Interior &   12      &  $\sim$      \\ \hline
Human ID  &     122    & 1870           & Exterior &   2      &$\sim$       \\ \hline
HumanEva &     4+2    & 56           & Interior &   4/7      &  Si       \\ \hline
INRIA \footnote{INRIA Perception, Multicam Dataset. } &       >11    & >40           & Interior &   $\geq$1      &  Si       \\ \hline
CMU Mobo \footnote{CMU Motion of Body.} &     25    & 100           & Caminadora &   6      &  $\sim$       \\ \hline
CASIA &     385    & $\sim$           & Interior y  &   >4      &  $\sim$       \\ 
 &         &            & exterior  &         &      \\ \hline
MHAD \footnote{Berkeley Multimodal Human Action Database.} & 12         & 660            & interior  & 12        & Si      \\ 
\hline \hline


\rowcolor[HTML]{CBCEFB}
\textbf{Base}     & \textbf{Movimientos}  & \textbf{Apariencia}    & \textbf{Ground Truth} & & \\
\rowcolor[HTML]{CBCEFB}
\textbf{de datos} & \textbf{disponibles} &               &           & & \\
\hline \hline

{M.C.L. }   & Varios    &  Traje MoCap & 3D-Vicon \footnote{ Esta notación indica que la captura considerada ground truth, se hizo con un sistema  de captura de movimiento (MoCap) de Vicon-Peak, \textcolor{blue}{\underline{\url{http://www.vicon.com/}}} } & & \\ \hline
{C.M.U }    & Varios    &  Traje MoCap & 3D-Vicon \footnote{Secuencias bvh de CMU, \textcolor{blue}{\underline{\url{http://sites.google.com/a/cgspeed.com/cgspeed/motion-capture}}}.} & & \\ \hline
{G.T} &     Marcha fuera&     Traje MoCap y       & 3D  en formato & & \\ 
	 &	   de régimen  &  Natural   &   Maya & & \\ \hline
U.S. &       Marcha    &  Natural    &  $\sim$ & &       \\	\hline
Human ID &     Marcha    & Natural  & $\sim$ &  &       \\ \hline
HumanEva &     Varios    & Natural  & 3D - Vicon & &   \\ \hline
INRIA &       Varios    & Traje MoCap y            & 2D - MoCap &   &       \\
  &           & Natural            &   etiquetado manual  & &         \\ \hline
  CMU Mobo &     5 tipos de marcha    & $\sim$           & Etiquetado Manual \footnote{Disponible en  \textcolor{blue}{\underline{\url{http://www.cs.cmu.edu/~zhangjy/\# Data}}}.}&         &        \\ \hline
CASIA &  Varias velocidades       &   Natural         & No  &         &      \\ 
 & de marcha        &            &   &         &      \\ \hline
MHAD & Varios        & Traje MoCap            & 3D - Impulse \footnote{ Esta notación indica que la captura considerada ground truth, se hizo con un sistema  de captura de movimiento (MoCap) de Vicon-Peak, \textcolor{blue}{\underline{\url{http://www.vicon.com/}}} }    &   &       \\
\hline

	\end{tabular}
	\end{minipage}
	
\end{table}

A continuación algunos comentarios que vale la pena resaltar sobre  las bases anteriores.


\paragraph{Motion Capture Lab.\footnote{\textcolor{blue}{\underline{\url{http://accad.osu.edu/research/mocap/mocap_home.htm}}}. Accedido 30-11-14}} 
		Como describe el nombre, este laboratorio se centra en obtener capturas de movimiento tridimensionales a través del sistema Vicon. Cuentan con un sistema Vicon 8i de 14 cámaras, dos videocámaras digitales Sony DSR-PD150 que llegan a 30 fps\footnote{\textcolor{blue}{\underline{\url{http://www.bhphotovideo.com/c/product/197878-REG/Sony_DSRPD150_DSR_PD150_Professional_1_3_DVCAM.html}}} ,Accedido 29-11-14 }, junto a un sistema de sincronización que permite integrar todas estas cámaras en el laboratorio. El video se utiliza únicamente como ayuda en los datos de captura para corregir y ver los marcadores,  por lo que \underline{no cuenta con secuencias de video adecuadas}. Cabe destacar que maneja múltiples formatos MoCap, inclusive el bvh. 



\paragraph{Carnegie Mellon University Motion Capture Database.\footnote{\textcolor{blue}{\underline{\url{http://mocap.cs.cmu.edu/}}}. Accedido 30-11-14}}

 Este laboratorio también se centra en obtener capturas de movimiento a través de un sistema Vicon, y no está implementado para evaluaciones ópticas de las capturas. Solo maneja videos monoculares de baja resolución, por lo que no cuenta con secuencias de video adecuadas para nuestro proyecto. Sin embargo maneja múltiples formatos MoCap y  posee herramientas para conversión a otros formatos, incluido el bvh. Posee descripción detallada de la ubicación de los marcadores, así como del laboratorio. El sujeto a relevar se encuentra vestido con ropas finas y ajustadas, sobre la cual se colocan los marcadores, por lo tanto se puede despreciar las fluctuaciones de posición de marcadores debidas a la ropa, que si exhiben otros ambientes menos controlados  Esta base de datos es bastante utilizada en el ámbito de la  animación por computadora y es por lejos la que dispone de mayor cantidad de capturas de movimiento de acceso público de las bases relevadas en esta sección.

\paragraph{Georgia Tech.\footnote{\textcolor{blue}{\underline{\url{http://www.cc.gatech.edu/cpl/projects/hid/index.html}}}. Accedido 30-11-14}}
Se encuentra desarrollando maneras de identificar a los seres humanos a distancia a través del reconocimiento de la marcha. También llevan a cabo trabajos relacionados en la localización y seguimiento de rostros, la detección de oclusiones y actividades específicas de sustracción de fondo.
Lamentablemente alguno las cámaras están todas sobre un costado del caminante, la resolución es baja y no todos los videos poseen marcadores. No se cuidan las condiciones de laboratorio para trabajar de manera óptica según nuestras necesidades.


\paragraph{University of Southampton Database.\footnote{ \textcolor{blue}{\underline{\url{http://www.gait.ecs.soton.ac.uk/database}}}. Accedido 30-11-14}}
Interesados en el reconocimiento de personas a través del seguimiento de la marcha, relevantes sobre todo por ser pioneros en el estudio biomecánico de la marcha. Poseen dos bases de datos, HiD gait database  (100 sujetos) y Biometric tunnel. El servidor donde se alojaba la base de datos está fuera de línea desde el 2004, el encargado de la página es Mark Nixon (\textcolor{blue}{\underline{\url{msn@ecs.soton.ac.uk}}}). Aparentemente se continúa actualmente el proyecto en \textcolor{blue}{\underline{\url{http://www.cspc.ecs.soton.ac.uk/gait}}}\footnote{Accedido 30-11-14}. Si bien estas bases de datos no están basadas en marcadores, cabe resaltar que el fondo del espacio de captura del túnel biométrico contiene patrones asimétricos con colores saturados que facilitan no solo la extracción de fondo sino también la automatización del proceso de calibración.

\paragraph{Human ID Gait Challenge Dataset.\footnote{\textcolor{blue}{\underline{\url{http://marathon.csee.usf.edu/GaitBaseline/ }}}. Accedido 30-11-14} }
Contiene 1.2 Tera bytes de información. Utilizada para el reconocimiento de la marcha,  no posee marcadores. Al igual que en la base de Southampton contiene en la secuencia imágenes de dameros convenientemente dispuestos, útiles para calibrar las cámaras. Las sucesivas secuencias tomadas para un mismo sujeto modifican distintos tipos de factores, como la calidad del terreno a recorrer, si el sujeto lleva o no un maletín y el  tipo de calzado utilizado. Tienen disponible junto a todas las secuencias de la base de datos las siluetas de los sujetos, obtenidas con un algoritmo base también propuesto y disponible. Lamentablemente el enlace que indica el protocolo seguido junto con la especificación del equipamiento está fuera de línea.

\paragraph{HumanEva.\footnote{\textcolor{blue}{\underline{\url{http://vision.cs.brown.edu/humaneva/index.html }}}. Accedido 30-11-14} \cite{humaneva} }
Con solo 13 Gigabit de información, el trabajo de este laboratorio es muy completo, incluye métricas de evaluación y un análisis importante de las necesidades actuales a la hora de generar una base de datos para relevar actividades humanas. El único inconveniente para cumplir los requisitos de nuestro proyecto es que no está pensado para el seguimiento óptico de marcadores, por lo que los marcadores sobre los sujetos de estudio utilizados para recabar datos Mocap son demasiado pequeños, las condiciones de luminosidad y color de fondo no son las adecuadas para nuestro propósito, incluso utilizan máscaras sobre las imágenes ópticas para cubrir los marcadores, estos últimos son utilizados solo para recopilar el ground truth a través de un sistema Vicon.  Dado que su motivación es obtener una imagen ``natural'' del sujeto que contenga la complejidad generada por el movimiento de la ropa, el ground truth que presentan  no es tan preciso como los obtenidos por métodos más tradicionales.  Contiene código para efectuar sustracción de fondo y la implementación de un algoritmo base, con un filtrado de partículas utilizado para el seguimiento de la pose del sujeto. Para acceder a los datos se debe gestionar un permiso en la página de HumanEva. 

\paragraph{INRIA Perception, Multicam Dataset.\footnote{INRIA Perception, Multicam Dataset, \textcolor{blue}{\underline{\url{http://4drepository.inrialpes.fr/pages/home  }}}. Accedido 30-11-14}} Efectúan las capturas con múltiples cámaras y también ofrecen la secuencia de malla de los sujetos reconstruidos a partir de las imágenes. Ofrecen un software que permite navegar en 4D, es decir, el espacio y el tiempo, sobre los modelos disponibles en su base de datos. Por lo que estos modelos se pueden ver desde cualquier ángulo de visión e inspeccionar congelándolos en cualquier instante de tiempo. Lamentablemente su captura no está basada en marcadores. 

\paragraph{CMU MoBo Dataset.\footnote{CMU Motion of Body,  \textcolor{blue}{\underline{\url{http://www.ri.cmu.edu/publication_view.html?pub_id=3904 }}}. Accedido 30-11-14}}
Estudia la marcha humana, enfocada en la identificación Biométrica de humanos a partir de sus características individuales. Los sujetos bajo estudio efectúan 4 tipos de caminata sobre una caminadora: marcha lenta, marcha rápida, marcha sobre plano inclinado y marcha sosteniendo un balón. Se utilizan 6 cámaras de alta resolución ubicadas alrededor de la caminadora que adquieren imagenes a 30 frames/s. Para acceder a la base de datos debe pedirse acceso comunicandose con \textcolor{blue}{\underline{\url{rgross@cs.cmu.edu }}}

\paragraph{CASIA Gait Database.\footnote{ \textcolor{blue}{\underline{\url{http://www.cbsr.ia.ac.cn/english/Gait\%20Databases.asp
 }}}. Accedido 30-11-14}} 
Contiene 4 bases de datos sobre la marcha, la primera se efectúa en exteriores y posee una sola cámara; la segunda es en interiores  y utiliza 11 cámaras sobre un lado del sujeto utilizada para extraer la silueta del sujeto bajo estudio; la tercera utiliza cámaras infrarrojas y cada sujeto efectúa tres tipo de marcha, lenta, rápida y con mochila; por último la cuarta combina la información de cámaras con una plataforma que registra la presión de la planta del pie a medida que el sujeto desarrolla el movimiento. Además de los archivos de vídeo, ofrecen las siluetas humanas extraídas de archivos de vídeo. Las capturas visuales no están basadas en marcadores.

\paragraph{Berkeley Multimodal Human Action Database (MHAD).\footnote{\textcolor{blue}{\underline{\url{http://tele-immersion.citris-uc.org/berkeley_mhad\#dl  }}}. Accedido 30-11-14}}
El objetivo de esta base de datos es reconocer el movimiento del cuerpo al realizar este distintas actividades. Tiene la información necesaria para efectuar sustracción de fondo, devuelven información desde múltiples fuentes, sistema con cámaras ópticas formado por 4 conjuntos de cámaras estéreo, dos sensores Kinect, 6 acelerómetros sobre el sujeto y 4 micrófonos a cada lado del espacio de captura.  El problema está en que los marcadores utilizados, únicamente recopilan información para el ground truth, con lo cual no se cuida de obtener una buena información óptica de los mismos.


TENGO QUE RESALTAR QUE NO SE ENCONTRARON BASES DE DATOS QUE CONTEMPLARAN NUESTRAS NECESIDADES. LAS DE EFECTUAR SEGUIMIENTO OPTICO DE MARCADORES.

TENGO QUE ACTUALIZAR LAS DESCRIPCIONES DE LAS BASES DESDE INRIA PARA ABAJO.
\subsubsection{Conclusiones}
Si bien la búsqueda se centró en blablabla no logró encontrar una base de datos que cumpliera las especificaciones de nuestro proyecto. Debido a blablabla.
 De todas maneras se genera un relevamiento de bases de datos para el movimiento humano, se profundiza en las características usuales que presentan dichas bases de datos y se logra reunir un conjunto de conceptos y herramientas necesarios para introducirnos en el tema. 



\subsection{Características de un Laboratorio para un sistema de captura}
\subsubsection{Variables}
-tipo de cámara, fps (vs resolución temporal y tipo de movimiento)  y resolución (vs resolución espacial de datos de salida)\\
-tipo de marcador\\
-posición de marcadores---modelado del cuerpo humano\\
-vestimenta\\
-posición y tipo de luz\\
-dimensiones del espacio de captura(relacionarlo con la posición de las cámaras y el tipo de movimiento)\\
-fondo del espacio de captura\\
-posición de cámaras (describir como depende del movimiento--marcha-rectilínea, movimiento libre--área circular)\\
-facilidades para calibración\\
-Sincronización\\
-Formatos Mocap (terminología, ventajas, desventajas, formatos usuales)
\\


\textbf{ Entrada de sincronización}\footnote{http://en.wikipedia.org/wiki/Genlock, Accedido 29-11-14}
El sincronizador (“Gen lock” –abreviatura del inglés “generator lock”) – es una forma de sincronizar varias videocámaras. Esto resulta esencial en las emisiones con conexiones en directo con varias cámaras, para garantizar que las imágenes de vídeo se mantienen estables incluso cuando se cambia de una cámara a otra.
Una señal de sincronización, enviada a todas las videocámaras que participan en una grabación desde una fuente central, indica a cada una de ellas cuando deberán comenzar a grabar. La misma orden es enviada para cada fotograma y todas y cada una de las líneas escaneadas dentro del fotograma. El sincronizador ‘bloquea’ las videocámaras en la misma secuencia de escaneado.
Si no se contara con una entrada de sincronización sería prácticamente imposible realizar una grabación con varias cámaras simultáneamente.


RECORDATORIO: Aumentar el número de cámaras permite una mayor cobertura de volumen, lo que significa una mayor visibilidad del conjunto de marcadores. A su vez una mejor visibilidad del conjunto de marcadores está correlacionado con disminuir la posibilidad de que los datos importantes se pierdan.




IR LLENANDO ESTA INFORMACIÓN ($2+2=4$)

\subsubsection{Conclusiones}
especificar los parámetros deseables en una base de datos que trabaje con información óptica basada en marcadores

\subsection{Base de datos sintética}
\subsubsection{Motivación}
\subsubsection{Implementación}
HABLAR SOBRE estructura de la base de datos\\
HABLAR SOBRE estructura de los datos donde se guarda la información.\\
\subsubsection{Conclusiones}
sobre como quedo la base de datos, utilidad, potencial y posible expansión

\subsection{Conclusiones}
sobre todo el capitulo, resumen de conclusiones parciales



