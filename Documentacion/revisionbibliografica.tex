\section{Revisión bibliográfica}
\label{invBiblio}

Como se menciona en la introducción, la primer etapa realizada en el proyecto fue la revisión bibliográfica de sistemas de captura de movimiento. Esta primera etapa permite obtener una idea de los distintos procesos que conforman un sistema de captura de movimiento. Estos son:

\begin{itemize}
	\item Calibración.
	\item Adquisición.
	\item Detección de marcadores (Segmentación).
	\item Seguimiento.
	\item Estimación de pose (Reconstrucción).
\end{itemize}

Al no encontrar detalles sobre el diseño de sistemas que verifiquen las hipótesis de este proyecto, o que puedan reutilizarse como base, se decide implementar uno de manera integra, realizando una búsqueda bibliográfica detallada y asignando cada artículo a una categoría de acuerdo a los procesos definidos anteriormente. Esta etapa del proyecto, por momentos tomó carácter de investigación científica más que de proyecto ingenieril. 
En el apéndice I (capítulo \ref{tablabiblio}), se muestra una tabla con toda la bibliografía relevante, ordenada según las categorías antes mencionadas.


En total se clasificaron 18 documentos, dentro de los cuales se encuentran artículos, papers, y publicaciones en general. Los más destacados dentro de cada grupo son:

\begin{itemize}
	\item Calibración.
	\begin{itemize}
		\item \emph{Optical Motion Capture System with Pan-Tilt Camera Tracking and Realtime Data Processing}.
	\end{itemize}
	\item Detección de marcadores (Segmentación).
	\begin{itemize}
		\item \emph{Simple and robust hard cut detection using interframe differences}. En esta investigación se propone un posible algoritmo de segmentación para utilizar en la etapa de detección de marcadores, utilizando un método de detección de bordes. Por motivos que se verán más adelante, estos tipos de algoritmos se descataron para utilizar en segmentación (\textit{ver sección \ref{deteccionMarcadoresSec}}).
		\item \emph{Threshold survey}\cite{surveyThreshold}. Se listan 40 algoritmos de segmentación mediante métodos de umbral. Se evalúan los mismos y se muestran ventajas y desventajas de cada uno. Entre ellos se encuentra el método de umbral de Otsu\cite{otsu}, el cual fue el elegido para implementar en este sistema (\textit{ver sección \ref{deteccionMarcadoresSec}}).
	\end{itemize}
	\item Seguimiento.
	\begin{itemize}
		\item \emph{Skeleton-Based Motion Capture for Robust Reconstruction of Human Motion}\cite{herda}.
		\item \emph{Simple and robust hard cut detection using interframe differences}.
		\item \emph{Modelling and Tracking Articulated Motion from Multiple Camera Views}.
		\item \emph{Skeletal Parameter Estimation from Optical Motion Capture Data}.
		\item \emph{Optical Motion Capture System with Pan-Tilt Camera Tracking and  Realtime Data Processing}.
		\item \emph{Resolving Motion Correspondence for Densely Moving Points}.
	\end{itemize}
	\item Estimación de pose (Reconstrucción).
	\begin{itemize}
		\item \emph{Skeleton-Based Motion Capture for Robust Reconstruction of Human Motion}\cite{herda}.
		\item \emph{Modelling and Tracking Articulated Motion from Multiple Camera Views}.
		\item \emph{Skeletal Parameter Estimation from Optical Motion Capture Data}.
		\item \emph{Optical Motion Capture System with Pan-Tilt Camera Tracking and  Realtime Data Processing}.
		\item \emph{What can two images tell us about a third one?}.
	\end{itemize}
	\item Sistemas completos
	\begin{itemize}
		\item \emph{Detección, rastreo y reconstrucción tridimensional de marcadores pasivos para análisis de movimiento humano}.
		\item \emph{Marker Detection and trayectory generation algorithms for a multicamera based gait analysis system}.
		\item \emph{Analisis de video para estimacion del movimiento humano}.
	\end{itemize}
\end{itemize}

Se puede ver, que en varios casos un mismo artículo contiene información de valor sobre varias etapas del sistema. Además se encontraron algunos documentos con diseños de sistemas completos.

A partir de esta etapa se definieron ciertos aspectos de vital importancia para el proyecto, en particular, se pudo seleccionar para el bloque de detección de marcadores el método de segmentación por umbral dinámico de Otsu\cite{otsu}. Este método fue elegido por tener el mejor compromiso entre simplicidad y eficacia.

En los bloques de seguimiento y estimación de pose se decide basar el desarrollo en las ideas del sistema propuesto por Lorna Herda en sus tesis de doctorado\cite{herda}. Dicha elección se debe a que el estudio de Herda contiene las ideas principales para implementar un sistema de captura de movimiento de manera compacta y general, posee una gran cantidad de menciones siendo un referente en el tema y la documentación disponible es amplia (dos papers, y tesis completa). Por otro lado, dicho trabajo se encuentra bajo las mismas hipótesis de uso que el estudio preliminar realizado por este equipo de proyecto (uso en fisioterapia, animación, biomecánica, entrenamiento en alto rendimiento), y es un punto de referencia habitualmente mencionado. Sin embargo, si bien en todas sus menciones la metodología de tratamiento de datos es la misma, tanto los bloques de adquisición como de detección van variando en equipamiento y métodos en los distintos proyectos.