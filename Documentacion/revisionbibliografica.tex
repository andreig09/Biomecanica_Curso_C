\section{Revisión bibliográfica}
\label{invBiblio}

Como se mencionó en la introducción, la primer etapa a realizarse en el proyecto fue una revisión bibliográfica sobre los sistemas de captura de movimiento. Con esta primera investigación se logró obtener una idea de los diferentes procesos que componen un sistema de captura de movimiento. Estos son:

\begin{itemize}
	\item Calibración.
	\item Adquisición.
	\item Detección de marcadores (Segmentación).
	\item Seguimiento.
	\item Estimación de pose (Reconstrucción).
\end{itemize}

Dado que no se encontraron diseños de sistemas completos que verifiquen las hipótesis de este proyecto, o que puedan utilizarse como base para realizarlo, se decidió implementar uno. A partir de esto se realizó una búsqueda bibliográfica detallada, asignando cada artículo a una categoría de acuerdo a los procesos definidos anteriormente. Esta etapa del proyecto, por momentos tomó un carácter más apropiado para una investigación científica que para un proyecto ingenieril. 

En el apéndice I (capítulo \ref{tablabiblio}), se muestra una tabla con la clasificación realizada.


En total se clasificaron 18 documentos, dentro de los que se encuentran artículos, papers, y publicaciones en general. Entre ellos, se pueden destacar en cada grupo de la clasificación:

\begin{itemize}
	\item Calibración.
	\begin{itemize}
		\item \emph{Optical Motion Capture System with Pan-Tilt Camera Tracking and Realtime Data Processing}.
	\end{itemize}
	\item Detección de marcadores (Segmentación).
	\begin{itemize}
		\item \emph{Simple and robust hard cut detection using interframe differences}. En esta investigación se propone un posible algoritmo de segmentación para utilizar en la etapa de detección de marcadores, utilizando un método de detección de bordes. Por motivos que se verán más adelante, estos tipos de algoritmos se descataron para utilizar en segmentación (\textit{ver sección \ref{deteccionMarcadoresSec}}).
		\item \emph{Threshold survey}\cite{surveyThreshold}. Se listan 40 algoritmos de segmentación mediante métodos de umbral. Se evalúan los mismos y se muestran ventajas y desventajas de cada uno. Entre ellos se encuentra el método de umbral de Otsu\cite{otsu}, el cual fue el elegido para implementar en este sistema (\textit{ver sección \ref{deteccionMarcadoresSec}}).
	\end{itemize}
	\item Seguimiento.
	\begin{itemize}
		\item \emph{Skeleton-Based Motion Capture for Robust Reconstruction of Human Motion}\cite{herda}.
		\item \emph{Simple and robust hard cut detection using interframe differences}.
		\item \emph{Modelling and Tracking Articulated Motion from Multiple Camera Views}.
		\item \emph{Skeletal Parameter Estimation from Optical Motion Capture Data}.
		\item \emph{Optical Motion Capture System with Pan-Tilt Camera Tracking and  Realtime Data Processing}.
		\item \emph{Resolving Motion Correspondence for Densely Moving Points}.
	\end{itemize}
	\item Estimación de pose (Reconstrucción).
	\begin{itemize}
		\item \emph{Skeleton-Based Motion Capture for Robust Reconstruction of Human Motion}\cite{herda}.
		\item \emph{Modelling and Tracking Articulated Motion from Multiple Camera Views}.
		\item \emph{Skeletal Parameter Estimation from Optical Motion Capture Data}.
		\item \emph{Optical Motion Capture System with Pan-Tilt Camera Tracking and  Realtime Data Processing}.
		\item \emph{What can two images tell us about a third one?}.
	\end{itemize}
	\item Sistemas completos
	\begin{itemize}
		\item \emph{Detección, rastreo y reconstrucción tridimensional de marcadores pasivos para análisis de movimiento humano}.
		\item \emph{Marker Detection and trayectory generation algorithms for a multicamera based gait analysis system}.
		\item \emph{Analisis de video para estimacion del movimiento humano}.
	\end{itemize}
\end{itemize}

Se puede ver, que en varios casos un mismo artículo contiene información de valor sobre varias etapas del sistema. Además se encontraron algunos documentos con diseños de sistemas completos.

A partir de esta etapa se definieron ciertos aspectos de vital importancia para el proyecto, en particular, se pudo seleccionar para el bloque de detección de marcadores el método de segmentación por umbral dinámico de Otsu\cite{otsu}. Este método fue elegido por tener el mejor compromiso entre simplicidad y eficacia.

Para los bloques de seguimiento y estimación de pose de se decidió implementar el sistema propuesto por Lorna Herda en sus tesis de doctorado\cite{herda}. Se decidió por esta metodología por ser aquella que presenta todo lo necesario para un sistema de captura de movimiento de forma más compacta, con mayor cantidad de menciones y amplia documentación (dos papers, y tesis completa). Por otro lado, dicho trabajo se encuentra bajo las mismas hipótesis de uso que el estudio preliminar realizado por este equipo de proyecto (uso en fisioterapia, animación, biomecánica, entrenamiento en alto rendimiento), y es un punto de referencia habitualmente mencionado. Sin embargo, si bien en todas sus menciones la metodología de tratamiento de datos es la misma, tanto los bloques de adquisición como de detección van variando en equipamiento y métodos en los distintos proyectos.