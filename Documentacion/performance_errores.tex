\section{Evaluación}

\subsection{Medida de Error}

Tanto para el caso de la segmentación de los vídeos como la reconstrucción de los marcadores, se tiene los verdaderos marcadores generados en la secuencia, por lo que es necesario establecer una medida de error sobre la salida de los algoritmos. Se eligió aplicar una metodología basada en la distancia euclidiana promedio, entre los marcadores de ambos conjuntos \cite{humaneva} . 

Asumiendo que los conjuntos de datos de salida de algoritmos y ground truth, tienen sus cuadros sincronizados temporalmente, en un cuadro [f] dado se tienen $M$ marcadores en un conjunto $ \boldsymbol{x},\quad\{m_{i}(x)\},i=1,\ldots,M $ , donde $ m_{i}(x)\in{\mathbf{R}^{3}} $ ( o $ \mathbf{R}^{2} $ para el caso de las vistas de cámaras individuales), y $ \boldsymbol{\tilde{x}} $ es el conjunto ground truth con la misma cantidad $M$ de marcadores alineados con $\boldsymbol{x}$ (el $i$-esimo marcador de $\boldsymbol{x}$ se corresponde con el  $i$-esimo marcador de $\boldsymbol{\tilde{x}}$), se calcula la distancia como

\begin{equation}
D(\boldsymbol{x_{f}},\boldsymbol{\tilde{x_{f}}})=\frac{1}{M}\sum_{i=1}^{M} \|m_{i}^{[f]}(\boldsymbol{x})-m_{i}^{[f]}(\boldsymbol{\tilde{x}})\|
\label{error_vs_ground_basica}
\end{equation}

Sin embargo, el calculo debe ser modificado para contemplar para el caso en que la cantidad de marcadores a la salida del procesamiento es menor a $M$, se define un conjunto binario en cada frame $\Delta=\{\delta_1,\delta_2,\ldots,\delta_M\}$, donde $\delta_i$ indica con 1 si el $i$-esimo marcador de $\boldsymbol{\tilde{x}}$ fue detectado, 0 en caso contrario. La ecuación \ref{error_vs_ground_basica} queda entonces

\begin{equation}
D(\boldsymbol{x_{f}},\boldsymbol{\tilde{x_{f}}},\Delta_{f})=\frac{1}{\sum_{j=1}^{M} \delta_j} \sum_{i=1}^{M} \delta_i.\|m_{i}^{[f]}(\boldsymbol{x})-m_{i}^{[f]}(\boldsymbol{\tilde{x}})\|
\label{error_vs_ground_deteccion}
\end{equation}

En una secuencia con $F$ frames, la performance promedio es calculada como el promedio de los errores en cada frame,

\begin{equation}
\mu_{secuencia} = \frac{1}{F}\sum_{f=1}^{F} D(\boldsymbol{x_{f}},\boldsymbol{\tilde{x_{f}}},\Delta_{f})
\label{performance_secuencia}
\end{equation}

Para poder trabajar con los datos a la salida de la segmentación y reconstrucción de las secuencias, es necesario obtener datos adicionales y entonces aplicar la ecuación \ref{performance_secuencia}. Específicamente, las parejas entre marcadores obtenidos y aquellos en el ground truth , que permite alinear y comparar los marcadores, y definir cuales fueron detectados.

El emparejamiento es realizado frame a frame, calculando en una matriz la distancia euclidiana entre todos los pares $\{i,j\}$, donde $i=1,\ldots,M_{x}$ es un marcador del conjunto $\boldsymbol{x}$ obtenido mediante algoritmo en un frame [f], y $j=1,\ldots,M_{\tilde{x}}$ es un marcador del conjunto $\boldsymbol{\tilde{x}}$ de ground truth,

\begin{equation}
d_{i,j}^{[f]} = \{\|m_{i}^{[f]}(\boldsymbol{x})-|m_{j}^{[f]}(\boldsymbol{\tilde{x}})\|\}
\label{distancia_algoritmo_ground}
\end{equation}

Una vez calculadas todas las distancias para un frame, se buscan aquellas parejas que presenta la menor distancia, relevando la pareja $(i,j)$  para la cual se verifica. Una vez obtenida esta distancia, los marcadores $(i,j)$ quedan descartados de la matriz, volviendo a buscar la siguiente pareja con distancia mínima, hasta que ya no queden elementos para emparejar (lo cual sucede en caso que se generen menos marcadores que en el ground truth como puede suceder en segmentación, o si todos los marcadores de ground truth ya fueron emparejados y sobran marcadores, como sucede en reconstrucción). Trabajar con las parejas en todos los grames de la secuencia, es análogo a trabajar con la ecuación (\ref{error_vs_ground_deteccion}) .

\subsection{Performance}

\subsubsection{Capturas Sintéticas}

Resultados múltiples secuencias

\begin{table}[h]
\begin{tabular}{cccc|c|c|c|c|c|c|ll}
\cline{5-10}
                              &                                 &                             &         & SEGM.                                                   & SEGM.                                               & RECONS.                                                 & RECONS.                                             & TRACK.                                                  & TRACK.                                              &  &  \\ \cline{1-10}
\multicolumn{1}{|c|}{captura} & \multicolumn{1}{c|}{markers} & \multicolumn{1}{c|}{frames} & n.cams & \begin{tabular}[c]{@{}c@{}}Promedio\\ (px)\end{tabular} & \begin{tabular}[c]{@{}c@{}}99\%\\ (px)\end{tabular} & \begin{tabular}[c]{@{}c@{}}Promedio\\ (cm)\end{tabular} & \begin{tabular}[c]{@{}c@{}}99\%\\ (cm)\end{tabular} & \begin{tabular}[c]{@{}c@{}}Promedio\\ (cm)\end{tabular} & \begin{tabular}[c]{@{}c@{}}99\%\\ (cm)\end{tabular} &  &  \\ \cline{1-10}
\multicolumn{1}{|c|}{8,03,1}  & \multicolumn{1}{c|}{14}         & \multicolumn{1}{c|}{89}     & 17      & 1,1063                                                  & 3,6783                                              & 0,40778                                                 & 2,6384                                              & 0,4318                                                  & 2,9039                                              &  &  \\ \cline{1-10}
\multicolumn{1}{|c|}{8,07,1}  & \multicolumn{1}{c|}{14}         & \multicolumn{1}{c|}{62}     & 17      & 1,0871                                                  & 3,1172                                              & 0,34382                                                 & 1,806                                               & 0,34451                                                 & 1,806                                               &  &  \\ \cline{1-10}
\multicolumn{1}{|c|}{8,07,2}  & \multicolumn{1}{c|}{14}         & \multicolumn{1}{c|}{123}    & 17      & 1,0912                                                  & 3,3097                                              & 0,38736                                                 & 1,2681
& 0,36381                                                 & 1,2681                                              &  &  \\ \cline{1-10}
\multicolumn{1}{|c|}{8,11}    & \multicolumn{1}{c|}{13}         & \multicolumn{1}{c|}{94}     & 17      & 1,074                                                   & 2,9042                                              & 0,38836                                                 & 2,8705                                              & 0,38836                                                 & 2,8705                                              &  &  \\ \cline{1-10}
\multicolumn{1}{|c|}{9,07,1}  & \multicolumn{1}{c|}{14}         & \multicolumn{1}{c|}{29}     & 17      & 1,1285                                                  & 3,3211                                              & 0,28016                                                 & 2,0401                                              & 0,28016                                                 & 2,0401                                              &  &  \\ \cline{1-10}
\multicolumn{1}{|c|}{9,07,2}  & \multicolumn{1}{c|}{14}         & \multicolumn{1}{c|}{57}     & 17      & 1,1404                                                  & 3,4476                                              & 0,34517                                                 & 1,9902                                              & 0,34517                                                 & 1,9902                                              &  &  \\ \cline{1-10}
\multicolumn{1}{|c|}{9,12,1}  & \multicolumn{1}{c|}{13}         & \multicolumn{1}{c|}{300}    & 15      & 1,0443                                                  & 2,118                                               & 0,39731                                                 & 1,4952                                              & 0,39741                                                 & 1,4952                                              &  &  \\ \cline{1-10}
\end{tabular}
\caption{Resultados de los bloques, para distintas capturas sintéticas}
\label{resultados_distintas_capturas}
\end{table}

\subsubsection{Ruido En Segmentación}

Resultado Reconstrucción (cm) ,ingresando Cam + ruido en pixel

\begin{table}[h]
\begin{tabular}{|c|c|c|c|c|c|}
\hline
\textbf{Caso} & \textbf{\begin{tabular}[c]{@{}c@{}}Ruido\\ (px)\end{tabular}} & \textbf{\begin{tabular}[c]{@{}c@{}}14\\ Reconstr.\end{tabular}} & \textbf{\begin{tabular}[c]{@{}c@{}}18\\ Reconstr.\end{tabular}} & \textbf{\begin{tabular}[c]{@{}c@{}}30\\ Reconstr.\end{tabular}} & \textbf{\begin{tabular}[c]{@{}c@{}}46\\ Reconstr.\end{tabular}} \\ \hline
Ground Cam    & 0                                                             & 7,84E-07                                                        & 7,84E-07                                                        & 7,84E-07                                                        & 7,84E-07                                                        \\ \hline
Segmentación  & 0                                                             & 1,5792                                                          & 0,38556                                                         & 0,38556                                                         & 0,38556                                                         \\ \hline
Ground Cam    & 0,5                                                           & 4,0817                                                          & 0,45675                                                         & 0,4453                                                          & 0,42607                                                         \\ \hline
Segmentación  & 0,5                                                           & 13,6991                                                         & 0,61128                                                         & 0,61381                                                         & 0,60517                                                         \\ \hline
Ground Cam    & 1                                                             & 16,9275                                                         & 0,82309                                                         & 0,72011                                                         & 0,73301                                                         \\ \hline
Segmentación  & 1                                                             & 11,0847                                                         & 0,8557                                                          & 0,82286                                                         & 0,83073                                                         \\ \hline
Ground Cam    & 2                                                             & 19,7131                                                         & 1,7649                                                          & 1,2152                                                          & 1,2027                                                          \\ \hline
Segmentación  & 2                                                             & 15,9698                                                         & 1,9701                                                          & 1,3021                                                          & 1,2884                                                          \\ \hline
Ground Cam    & 4                                                             & 28,6943                                                         & 6,1386                                                          & 2,1877                                                          & 2,0468                                                          \\ \hline
Segmentación  & 4                                                             & 51,4025                                                         & 5,7738                                                          & 2,1484                                                          & 2,1335                                                          \\ \hline
Ground Cam    & 8                                                             & 53,9555                                                         & 14,3684                                                         & 5,4929                                                          & 3,8377                                                          \\ \hline
Segmentación  & 8                                                             & 264,8935                                                        & 13,1539                                                         & 4,7142                                                          & 3,7993                                                          \\ \hline
\end{tabular}
\caption{Resultados de Error Promedio (cm) en reconstrucción, para distintos caso de ruido en la segmentación, tanto resultados sobre vídeos como sobre ground truth}
\end{table}

Mas ruido implica mas incertidumbre en la reconstrucción, mientras mas ruidos, mas marcadores reconstruidos se precisan en reconstrucción para compensar la precision. Para mayor cantidad de marcadores en reconstrucción, se hace inviable la identificación de marcadores en tracking ?

\subsubsection{Variación Cantidad de Cámaras}

Reconstrucción con menos cámaras

Con menos cámaras, según que movimiento se está capturando indica que marcadores se ven afectados en reconstrucción


Comportamiento caso real?


