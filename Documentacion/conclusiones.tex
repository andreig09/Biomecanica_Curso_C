\section{Conclusiones}
%conclusiones, pendientes, etc

En esta sección se evaluarán las conclusiones obtenidas a partir de lo realizado en el proyecto. Las mismas se pueden separar en conclusiones generales, y en particulares de los capítulos vistos anteriormente.

Como conclusiones generales se presentan las siguientes:
\begin{itemize}
\item Se cumplió el objetivo principal: se logró obtener un sistema completo que a partir de las capturas de video de una persona en un ambiente de laboratorio con las condiciones adecuadas, obtiene la posición 3D de  los marcadores presentes en el cuerpo de dicha persona, logrando representar su movimiento con una precisión del orden del centímetro.
\item Se logró relevar la literatura existente, y se realizó una clasificación de los documentos de acuerdo a la relevancia que prestan para construir un sistema de estas características. 
\item Se relevó el software existente, y se buscaron implementaciones en matlab, C/C++ y otros lenguajes, pero no se encontró un software Open Source que se adaptara a las características necesarias para utilizar de base en este proyecto. Debido a esto, se decidió implementar los bloques por cuenta propia, utilizando algunos algoritmos básicos de procesamiento de imágenes ya implementados o realizados específicamente para este proyecto. Esta etapa por momentos tuvo un carácter más de investigación científica que de proyecto ingenieril.
\item Se identificaron los distintos casos de uso, y se decidió implementar el sistema en base al caso de uso de marcha.
\item Se lograron implementar los 4 grupos de algoritmos principales que componen la aplicación: calibración, detección de marcadores, estimación de pose (o reconstrucción) y seguimiento.
\item Se generó un sistema que integra los 4 bloques nombrados en el punto anterior, y que dada la entrada definida obtiene la salida esperada.
\item Se elaboró una interfaz gráfica de usuario (GUI) básica para que el sistema pueda ser utilizado de una forma más práctica y amigable por el usuario.
\item Se buscaron bases de datos que se encuentren dentro de las hipótesis del problema, de forma tal de probar el mismo para diferentes casos y evaluar su rendimiento. Debido a que no se encontró ninguna base de datos disponible que cumpla con las características deseadas, se elaboró una en base a secuencias sintéticas generadas en Blender.
\item Se generó un conjunto de benchmarks\footnote{Herramientas de evaluación de performance.}, capaces de obtener una buena medida del rendimiento del sistema. 
\end{itemize}

A continuación, se plantean las conclusiones específicas de las etapas principales del proyecto:
\begin{itemize}
	\item Relevamiento bibliográfico
	\begin{itemize}
		\item No se encontraron implementaciones de sistemas de captura de movimiento Open Source con las características del sistema de este proyecto. 
	\end{itemize}
	\item Base de datos
	\item Calibración
	\item Detección de marcadores
	\begin{itemize}
		\item Se logró implementar un conjunto de algoritmos capaces de detectar la posición 2D de los marcadores en el cuerpo del paciente para todo instante de tiempo y exponer los resultados en un archivo XML. 
		\item Los resultados correspondientes a las pruebas con secuencias sintéticas fueron muy buenos, logrando un error que en la mayoría de los casos se ubicó por debajo de los 4 píxeles.
		\item Por el lado del procesamiento de secuencias reales, no se consiguieron secuencias que estén exactamente en las hipótesis del problema. Aún así, se realizaron pruebas con secuencias que se asemejaban a las condiciones necesarias obteniendo buenos resultados al extraer el fondo de la captura.
		\item Se observó que para ambientes donde las condiciones se mantienen constantes a lo largo del tiempo (como por ejemplo en un laboratorio), el umbral no cambia significativamente su valor, pudiéndose utilizar un valor constante en el para toda una secuencia. Esto permite ahorrar tiempo de procesamiento y reduce el costo computacional de la detección.
		\item Se observó que se puede mejorar la detección, ajustando los valores de las constantes $A$ y $B$ del filtro circular (ver sección \ref{implementSegment}) de forma tal de hacer el filtro más selectivo o menos.
	\end{itemize}
	\item Reconstrucción
	\item Seguimiento
	
	\begin{itemize}
	
	\item Los puntos reconstruidos en la reconstrucción son correctamente identificados y etiquetados en trayectorias, mediante la implementacion de un algoritmo simple de seguimiento combinando búsqueda de marcadores mas próximos con una estimación lineal determinada por la información previa de trayectoria	hasta el momento.
	
	\item Las perdidas de marcadores en pocos frames son detectadas, y estimadas al recuperar la trayectoria mediante predicciones lineales y radiales. Los marcadores recuperados cumplen las ecuaciones de continuidad de velocidad y aceleración.
	
	\item Las trayectorias obtenidas son evaluadas para encontrar grandes saltos en aceleración, detectar marcadores a corregir, e implementando correcciones que cumplan las hipótesis establecidas en el movimiento.
	
	\end{itemize}		
	
	\item Medida de performance
	\begin{itemize}
		\item Se obtuvo una precisión en la detección de los marcadores, con errores del orden centímetro.
		\item El máximo error por marcador no superó los 5 cm.
		\item Se obtuvieron muy buenos resultados, logrando reconstruir todos los marcadores en toda la secuencia hasta con un mínimo de 6 cámaras. Con menos cámaras se pierden marcadores ocasionalmente.
		\item Para el caso de 6 cámaras, la correcta detección del movimiento depende de la posición y la disposición entre ellas que tengan en el espacio.
	\end{itemize}
\end{itemize}

Por temas de cronograma y de presupuesto, en muchos aspectos no se utilizó la tecnología más moderna (por ejemplo sensores infrarrojos, cámaras de mejor calidad, etc.) que de haberlas utilizado se hubiesen logrado mejores resultados. Si bien esta primera versión del sistema no está a la altura de los sistemas comerciales vistos en este documento, se logró dar el primer paso y se dejaron todas las condiciones necesarias para poder continuar con el proyecto y llevar el mismo al nivel de las otras herramientas.

Quedan como tareas pendientes para etapas futuras:
\begin{itemize}
	\item Probar el sistema en un laboratorio con las características definidas en este proyecto y con 6 o más cámaras previamente calibradas.
	\item Adaptar el sistema para otros casos de uso además de la marcha.
	\item Robustecer la segmentación, ya sea agregando algoritmos que complementen la umbralización de Otsu\cite{otsu} o cambiando la umbralización por un algoritmo más complejo.
	
	\item Implementar un algoritmo alternativo de reconstrucción que funcione de forma mas robusta para el caso real
	
	\item Probar otros algoritmos de seguimiento de marcadores, e implementar restricciones de esqueleto y movimiento angular
	
	\item Establecer la realimentacion entre el bloque de reconstrucción y seguimiento para robustecer ambos bloques 	
	
	\item Mejorar la interfaz gráfica de usuario, haciéndola más amigable e intuitiva, de forma tal que el especialista pueda utilizarla como una aplicación de usuario.
	\item Realizar un manual de usuario.
\end{itemize}