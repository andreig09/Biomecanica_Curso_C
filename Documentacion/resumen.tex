El objetivo de este proyecto fue desarrollar un sistema de captura de movimiento para facilitar la tarea del análisis biomecánico del movimiento de las personas. La propuesta inicial fue realizada por Patricia Polero, investigadora del Departamento de Biofísica de la Facultad de Medicina de la Universidad de la República, Uruguay, en busca de una herramienta Open Source que le permita obtener datos y estadísticas específicas que las herramientas existentes no pueden ofrecer.

\vspace{3 mm}

Se elaboró un software con los bloques fundamentales que componen un sistema de estas características, utilizando los lenguajes \emph{C/C++}, \emph{Python} y \emph{Matlab}. Estos bloques son independientes unos de otros, lo que da la posibilidad de modificarlos o sustituirlos sin afectar el resto del sistema.

\vspace{3 mm}

También se creó un prototipo de base de datos, con secuencias de videos sintéticos, y un conjunto de algoritmos para medir la performance de cada bloque y del sistema en su totalidad.

\vspace{3 mm}

Las pruebas realizadas sobre el software implementado reflejaron que el mismo tiene una precisión del orden del centímetro. Estos resultados son buenos para ser una primera versión y teniendo en cuenta que los algoritmos utilizados en cada bloque son de complejidad baja y se pueden optimizar en todos sus aspectos.
